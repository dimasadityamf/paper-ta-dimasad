% Ubah judul dan label berikut sesuai dengan yang diinginkan.
\section{Pendahuluan}
\label{sec:pendahuluan}

% Ubah paragraf-paragraf pada bagian ini sesuai dengan yang diinginkan.

Penyakit tidak menular (PTM) meliputi penyakit kardiovaskular, kanker, penyakit pernapasan kronis dan diabetes telah menyebabkan kematian terhadap 41 juta orang setiap tahun dengan angka 74\% setara dengan nilai dari semua kematian secara global dan termasuk dalam penyebab utama kematian dan kecacatan dini. Indonesia memiliki angka kematian yang disebabkan oleh penyakit tidak menular sebesar 76\% dengan jumlah angka kematian sebanyak hampir 1,4 juta dan probabilitas kematian dini sekitar 25\% [1]. Faktor risiko utama dari terdiagnosis oleh penyakit tidak menular (PTM) adalah obesitas menyebabkan penurunan harapan hidup sekitar 5-20 tahun dengan melihat pada tingkat gangguan kondisi dan komorbid yang diderita [2]. Obesitas menjadi perhatian prioritas kesehatan global dalam mengatasi peningkatan tingkat kelebihat berat badan dan obesitas secara dini terhadap anak-anak dan remaja [3]. Obesitas dan penyakit gaya hidup menjadi perhatian dan masalah secara global [4].

Obesitas merupakan keadaan dimana terdapat akumulasi penumpukan lemak secara abnormal pada tubuh seseorang yang berlebihan dan menyebabkan berat badan pada nilai di atas normal yang dapat mengganggu kesehatan [5]. Indikasi yang dapat digunakan untuk menilai jika seseorang menderita kelebihan berat badan dan obesitas berdasarkan nilai body mass index (BMI) sebagai indek perbandingan berat badan dalam kilogram dengan kuadrat tinggi badan dalam meter [6]. Secara global nilai dari body mass index (BMI) memiliki tingkatan untuk kelebihan berat badan ditunjukkan dengan nilai lebih dari 25kg/m2 dan untuk tingkat obesitas ditunjukkan dengan nilai 30kg/m2 [7]. Keseimbangan energi dalam tubuh sangat penting karena penyebab obesitas dipengaruhi oleh kalori yang dikonsumsi tidak seimbang dengan kalori yang digunakan oleh tubuh [8]. Tingkat kebugaran yang rendah dan ketidakaktifan fisik dapat berdampak buruk pada kesehatan dan dapat menyebabkan penyakit kronis seperti diabetes dan penyakit kardiovaskular [9]. Salah satu yang dapat digunakan untuk mencegah obesitas dan meningkatkan kebugaran untuk mengurangi kelebihan berat badan adalah dengan melakukan olahraga.

Olahraga merupakan bentuk kegiatan kebugaran yang dilakukan sebagai kemampuan seseorang dalam memenuhi tuntutan fisik melalui kegiatan jasmani yang dilakukan secara terstruktur melibatkan pergerakan tubuh secara berulang-ulang [10]. Aktivitas fisik dengan melakukan olahraga dapat membantu dalam penurunan berat badan dan mengobati obesitas. Hubungan antara aktivitas fisik dan hasil kesehatan memiliki keterkaitan untuk meningkatkan kardiorespirasi dan kebugaran otot [11]. Kegiatan yang melibatkan aktivitas fisik dapat meningkatkan kesehatan pada individu dengan dapat meningkatkan kualitas hidup dari manfaat kesehatan jasmani pada tubuh [12]. Manfaat dari kesehatan jasmani dengan melakukan aktivitas fisik sangat membantu peluang untuk kesehatan fisik dan mental terhadap program pengobatan obesitas [13]. Aktivitas olahraga dinilai bermanfaat dan sesuai prosedur dengan melihat bagaimana kualitas olahraga yang dilakukan. Pengukuran kualitas aktivitas fisik dapat dilakukan berdasarkan jumlah energi yang dikeluarkan selama melakukan aktivitas fisik yang dengan International Physical Activity Questionnaire (IPAQ) telah digolongkan menjadi kategori rendah, sedang dan tinggi melihat nilai Metabolic Equivalent of Task yang dihasilkan dari perhitungan durasi dan frekuensi [14]. Jumlah energi yang dikeluarkan selama melakukan aktivitas olahraga akan berbeda-beda tergantung dari jenis aktivitas, durasi dan faktor fisik. Salah satu aktivitas olahraga yang dilakukan penelitian dalam meningkatkan kualitas aktivitas fisik untuk membantu penurunan berat badan dan obesitas adalah olahraga pada treadmill.

Treadmill digunakan dalam alat olahraga untuk dapat melakukan pergerakan tanpa berpindah tempat dan tidak dilakukan di atas tanah langsung. Alat ini juga digunakan dalam berbagai penelitian dan kebutuhan klinis [15]. Treadmil mudah digunakan karena dapat mengatur kontrol kecepatan pergerakan dengan melangkah yang dapat diatur dan membutuhkan sedikit ruangan [16]. Penggunaan treadmil terdapat berbagai jenis yang diantaranya dapat digunakan dengan kecepatan tetap dan kecepatan diri sendiri yang sangat bermanfaat untuk mengurangi neuromuskuler [17][18]. Berolahraga pada treadmill menghasilkan perbedaan yang tidak signifikan dalam hasil pembakaran energi dan kalori dengan melakukan olahraga tanpa treadmill yang membuat treadmill sangat digemari untuk digunakan sehari-hari [16].

Perkembangan teknologi pada pengembangan mengenai pengenalan aktivitas manusia mulai berkembang. Dalam beberapa tahun terakhir, pengenalan aktivitas manusia dengan menggunakan sensor pengukuran inersia berkembang dengan tanpa ada sensor melalui visi komputer yang lebih praktis [19]. Pengembangan ilmu visi komputer turut berkembang dengan pembelajaran mesin menggunakan \emph{deep learning} yang berkembang dalam visi komputer, pemrosesan bahasa maupun pengenalan gambar, video dan suara [20]. Munculnya ilmu \emph{deep learning} menjadi metode yang populer dalam melakukan pengenalan aktivitas yang dibantu dengan metode \emph{convolutional neural network} sebagai pembelajaran pengenalan aktivitas dengan menggunakan gambar berbabis pembelajaran mesin [21][22].

Aktivitas yang dilakukan pada treadmill dengan perhitungan pembakaran kalori hanya dapat dilakukan pada beberapa jenis treadmill yang memiliki sistem perhitungannya. Treadmill dengan sistem yang kompleks memungkinkan memiliki harga jual yang lebih tinggi dari treadmill yang sederhana. Sistem yang digunakan hanya bisa digunakan pada treadmill saja tanpa bisa terhubung satu sama lain antar alat. Hal ini membuat pengumpulan data dari setiap aktivitas yang dilakukan tidak tercatat dengan baik. Oleh karena itu, diperlukan sistem prediksi jumlah kalori yang terbakar yang lebih praktis dan mudah digunakan untuk berolahraga pada treadmill.


Pembahasan pada paper ini dimulai dengan pemaparan mengenai metode penelitian (Bagian \ref{sec:MetodePenelitian}).
Berdasarkan hal tersebut, kami menunjukkan penelitian dan pembahasan (Bagian \ref{sec:PenelitianPembahasan}).
Terakhir, didapatkan kesimpulan dari penelitian yang telah dilakukan (Bagian \ref{sec:kesimpulan}).
